% Robert Adams	02/07/2012	CS 311

\documentclass[letterpaper,10pt]{article} %twocolumn titlepage 
\usepackage{graphicx}
\usepackage{amssymb}
\usepackage{amsmath}
\usepackage{amsthm}

\usepackage{alltt}
\usepackage{float}
\usepackage{color}
\usepackage{url}

\usepackage{balance}
\usepackage[TABBOTCAP, tight]{subfigure}
\usepackage{enumitem}
\usepackage{pstricks, pst-node}


\usepackage{geometry}
\geometry{margin=1in, textheight=8.5in} %textwidth=6in

%random comment

\newcommand{\cred}[1]{{\color{red}#1}}
\newcommand{\cblue}[1]{{\color{blue}#1}}

\usepackage{hyperref}

\def\name{Robert M Adams}

%% The following metadata will show up in the PDF properties
\hypersetup{
  colorlinks = true,
  urlcolor = black,
  pdfauthor = {\name},
  pdfkeywords = {cs311 ``operating systems'' },
  pdftitle = {S 311 Project 4: Python Server, C Client},
  pdfsubject = {CS 311 Project 4},
  pdfpagemode = UseNone
}


\begin{document}
  \title{CS 311 Project 4: Python Server, C Client}
  \author{Robert Adams}
\maketitle


\section{Design Decisions}


	I decided to use the JSON format to send and receive messages between the programs.
    Although this was more work, it allows me more flexibility then say parsing a sting 
    and breaking it into variables by white space. 


	To prevent remote code from being executed I only parsed until a
    line ending was reached, then if the parsed content is not in 
    JSON format, the program throws an exception and exits.  
    In a program with more practical use it may be best to just 
    ignore anything not conforming to the format.
	

\section{Difficulties}


	Parsing a JSON string with C turned out to be quite the hassle.
    I was finally able to find a header only library and included it,
    but even getting that to work with its minimal examples and 
    documentation took lots of trial and error.


	Using ‘select’ properly was also a challenge. I was
    initially trying to accept every connection instead of registering
    each new connection as a list for select to iterate over. 


\begin{table}
\centering
    \begin{tabular}{|l|l|}
\hline
        Date                           & Description                                                                  \\ \hline
  Fri Mar 16 22:34:23 2012 -0700     & compute more or less receives kill singal and dies \\
 Fri Mar 16 18:51:32 2012 -0700      &    code for part2, compute working \\
Wed Mar 14 12:24:19 2012 -0700        & server now sends and recvs all messages correctly: fixed logic i \\ 
  Sun Mar 11 23:38:22 2012 -0700      & manage refactored, handles all req inputs  \\
 Sat Mar 10 21:44:55 2012 -0800       & client recieves and parses json   \\
 Sat Mar 10 15:15:17 2012 -0800       & added socket connect code to compute cleint and server connect   \\
Wed Mar 7 14:53:53 2012 -0800         & compute has brute.cpp code \\
  Sun Mar 4 19:46:31 2012 -0800       &  added simple server in python and python client \\
   Sun Mar 4 12:22:11 2012 -0800      & touch blank files \\

\hline
    \end{tabular}
\caption{Pulled from git commit log.}\label{commit-logs}
\end{table}


\end{document}
